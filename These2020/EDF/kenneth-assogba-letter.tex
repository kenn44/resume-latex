\documentclass[11pt,a4paper,sans]{moderncv}        % possible options include font size ('10pt', '11pt' and '12pt'), paper size ('a4paper', 'letterpaper', 'a5paper', 'legalpaper', 'executivepaper' and 'landscape') and font family ('sans' and 'roman')

% moderncv themes
\moderncvstyle{classic}                            % style options are 'casual' (default), 'classic', 'oldstyle' and 'banking'
\moderncvcolor{green}                              % color options 'blue' (default), 'orange', 'green', 'red', 'purple', 'grey' and 'black'
%\renewcommand{\familydefault}{\sfdefault}         % to set the default font; use '\sfdefault' for the default sans serif font, '\rmdefault' for the default roman one, or any tex font name
%\nopagenumbers{}                                  % uncomment to suppress automatic page numbering for CVs longer than one page

% character encoding
\usepackage[utf8]{inputenc}                       % if you are not using xelatex ou lualatex, replace by the encoding you are using
%\usepackage{CJKutf8}                              % if you need to use CJK to typeset your resume in Chinese, Japanese or Korean

% adjust the page margins
\usepackage[scale=0.75]{geometry}
%\setlength{\hintscolumnwidth}{3cm}                % if you want to change the width of the column with the dates
%\setlength{\makecvtitlenamewidth}{10cm}           % for the 'classic' style, if you want to force the width allocated to your name and avoid line breaks. be careful though, the length is normally calculated to avoid any overlap with your personal info; use this at your own typographical risks...

% personal data
\name{Kenneth}{Assogba}
\title{Candidature}                               % optional, remove / comment the line if not wanted
\address{22, Rue Charles de Gaulle}{91400 Orsay}{France}% optional, remove / comment the line if not wanted; the "postcode city" and and "country" arguments can be omitted or provided empty
\phone[mobile]{06~14~26~95~55}                   % optional, remove / comment the line if not wanted
%\phone[fixed]{+2~(345)~678~901}                    % optional, remove / comment the line if not wanted
%\phone[fax]{+3~(456)~789~012}                      % optional, remove / comment the line if not wanted
\email{kennethassogba@gmail.com}                               % optional, remove / comment the line if not wanted
%\homepage{www.johndoe.com}                         % optional, remove / comment the line if not wanted
%\extrainfo{additional information}                 % optional, remove / comment the line if not wanted
%\photo[64pt][0.4pt]{picture}                       % optional, remove / comment the line if not wanted; '64pt' is the height the picture must be resized to, 0.4pt is the thickness of the frame around it (put it to 0pt for no frame) and 'picture' is the name of the picture file
%\quote{Some quote}                                 % optional, remove / comment the line if not wanted

% to show numerical labels in the bibliography (default is to show no labels); only useful if you make citations in your resume
%\makeatletter
%\renewcommand*{\bibliographyitemlabel}{\@biblabel{\arabic{enumiv}}}
%\makeatother
%\renewcommand*{\bibliographyitemlabel}{[\arabic{enumiv}]}% CONSIDER REPLACING THE ABOVE BY THIS

% bibliography with mutiple entries
%\usepackage{multibib}
%\newcites{book,misc}{{Books},{Others}}
%----------------------------------------------------------------------------------
%            content
%----------------------------------------------------------------------------------
\begin{document}
%-----       letter       ---------------------------------------------------------
% recipient data
\recipient{Aux responsables du recrutement }{
	Atos SE\\
	%Département de simulation "Structures"\\
	78 - Les Clayes}
\date{Samedi, 16 Novembre 2019}
\opening{Mesdames, Messieurs,}
\closing{Veuillez agréer, Mesdames, Messieurs, l’expression de mes salutations distinguées,}
%\enclosure[Attached]{curriculum vit\ae{}}          % use an optional argument to use a string other than "Enclosure", or redefine \enclname
\makelettertitle

\textbf{Objet :} Quantum computing - développement et intégration logiciels

Actuellement étudiant en Master 2 Ingénierie Mathématique - Analyse Numérique, Calcul Scientifique option Mécanique à Sorbonne Université (ex Université Pierre et Marie Curie), je suis particulièrement intéressé par le sujet : "Quantum computing - développement et intégration logiciels" que vous proposez.

Atos, leader européen du cloud, de la cybersécurité et du supercalcul brille depuis plusieurs années par son activité de Recherche et Développement. Ainsi depuis 2016 vous avez lancé un programme de recherche, qui vise à développer une plateforme de simulation quantique permettant aux chercheurs de tester des algorithmes destinés aux futurs ordinateurs quantiques. Ce pôle de recherche me passionne réellement.

En effet, entre Mai et Octobre 2019, j’ai effectué un stage de recherche sur les Schémas monotones discrets pour l'équation de Schrödinger. Ce stage m'a permis de travailler sur la construction de nouvelles méthodes numériques pour le contrôle optimal en mécanique quantique. En outre, j’ai eu à construire et à implémenter des algorithmes pour le contrôle de réactions chimiques à l'échelle quantique.

Cette première expérience à la frontière entre sciences fondamentales et appliquées m’a donné le goût pour la recherche. Je souhaite ainsi, la poursuivre au cours du stage que vous proposez. J’ai ainsi hâte de pouvoir en discuter avec vous au cours d'un entretien.




\makeletterclosing

\end{document}


%% end of file `template.tex'.
%% Copyright 2006-2013 Xavier Danaux (xdanaux@gmail.com).
