%-------------------------
% Resume in Latex
% Author : Kenneth Assogba
% License : MIT
%------------------------

\documentclass[letterpaper,11pt]{article}

\usepackage{latexsym}
\usepackage[empty]{fullpage}
\usepackage{titlesec}
\usepackage{marvosym}
\usepackage[usenames,dvipsnames]{color}
\usepackage{verbatim}
\usepackage{enumitem}
\usepackage[hidelinks]{hyperref}
\usepackage{fancyhdr}
\usepackage[utf8]{inputenc}

\pagestyle{fancy}
\fancyhf{} % clear all header and footer fields
\fancyfoot{}
\renewcommand{\headrulewidth}{0pt}
\renewcommand{\footrulewidth}{0pt}

% Adjust margins
\addtolength{\oddsidemargin}{-0.5in}
\addtolength{\evensidemargin}{-0.5in}
\addtolength{\textwidth}{1in}
\addtolength{\topmargin}{-.5in}
\addtolength{\textheight}{1.0in}

\urlstyle{same}

\raggedbottom
\raggedright
\setlength{\tabcolsep}{0in}

% Sections formatting
\titleformat{\section}{
  \vspace{-4pt}\scshape\raggedright\large
}{}{0em}{}[\color{blue}\titlerule \vspace{-5pt}]

%-------------------------
% Custom commands
\newcommand{\resumeItem}[2]{
  \item\small{
    \textbf{#1}{: #2 \vspace{-2pt}}
  }
}

\newcommand{\resumeSubheading}[4]{
  \vspace{-1pt}\item
    \begin{tabular*}{0.97\textwidth}{l@{\extracolsep{\fill}}r}
      \textbf{#1} & #2 \\
      \textit{\small#3} & \textit{\small #4} \\
    \end{tabular*}\vspace{-5pt}
}

\newcommand{\resumeSubItem}[2]{\resumeItem{#1}{#2}\vspace{-4pt}}

\renewcommand{\labelitemii}{$\circ$}

\newcommand{\resumeSubHeadingListStart}{\begin{itemize}[leftmargin=*]}
\newcommand{\resumeSubHeadingListEnd}{\end{itemize}}
\newcommand{\resumeItemListStart}{\begin{itemize}}
\newcommand{\resumeItemListEnd}{\end{itemize}\vspace{-5pt}}

%-------------------------------------------
%%%%%%  CV STARTS HERE  %%%%%%%%%%%%%%%%%%%%%%%%%%%%


\begin{document}

%----------HEADING-----------------
\begin{tabular*}{\textwidth}{l@{\extracolsep{\fill}}r}
  \textbf{\href{https://kennethassogba.co/}{\Large Kenneth Assogba}} & {\color{blue} Email} : \href{mailto: kenneth.assogba@etu.upmc.fr}{kenneth.assogba@etu.upmc.fr}\\
  %\href{https://kennethassogba.co/}{https://kenneth.assogba.co} & Mobile : 06 14 26 95 55 \\
  Stage en Calcul scientifique à partir du 1er Avril 2020 & {\color{blue} Mobile} : 06 14 26 95 55 \\
  {\color{blue} Github} : \href{https://github.com/kenn44}{https://github.com/kenn44} & 24 ans (12/02/1995)
\end{tabular*}


%-----------EDUCATION-----------------
\section{\color{blue}Education}
  \resumeSubHeadingListStart
    \resumeSubheading
    %Calcul Scientifique
      {Master 2 Ingénierie Mathématique - Analyse Numérique - Mécanique}{Paris, France}
      {\textbf{Sorbonne Université}}{Sept. 2019 -- Present}
    \resumeSubheading
      {Master Mathématiques Fondamentales et Applications - EDP et Géométrie}{Dangbo, Benin}
      {Institut de Mathematiques et de Sciences Physiques}{Oct. 2017 -- Août 2019}
      %\resumeItemListStart
      %\resumeItem{Stage de fin d'études}
      %{. $\rhd$ \textbf{, : Octave}.}
      %\resumeItemListEnd
    \resumeSubheading
      {Licence Informatique - Option Architecture Logicielle}{Cotonou, Benin}
      {École Superieure de Gestion d'Informatique et des Sciences}{Oct. 2016 -- Dec. 2017}
    \resumeSubheading
      {Licence Mathematiques - Informatique}{Dangbo, Benin}
      {Institut de Mathematiques et de Sciences Physiques}{Oct. 2015 -- Jun. 2017}
    \resumeSubheading
      {Classes Préparatoires MPSI}{Dangbo, Benin}
      {Institut de Mathématiques et de Sciences Physiques}{Oct. 2013 -- Jun. 2015}
  \resumeSubHeadingListEnd


%-----------EXPERIENCE-----------------
\section{\color{blue}Experience}
  \resumeSubHeadingListStart
  
      \resumeSubheading
	  {Stagiare Assistant de Recherche}{Dangbo, Benin}
	  {Unité de Recherche en Mathématique et Physique Mathématique - IMSP}{Mai 2019 - Août 2019}
	  \resumeItemListStart
	  \item{Schémas monotones discrets pour l'équation de Schrödinger
	  \\\textbf{$\rhd$ \small{Étude de la littérature sur le Contrôle optimal en Mécanique quantique
	  \\$\rhd$ Construction de schémas monotones implicites et explicites
	  \\$\rhd$ Implémentation des algorithmes obtenus et simulations sous Octave}}}
	  \resumeItemListEnd

%    \resumeSubheading
%      {Artisan Développeur}{Cotonou, Benin}
%      {Solutis \href{https://solutis-it.com/}{(solutis-it.com)} }{Oct 2017 - Mars 2019}
%      \resumeItemListStart
%        \resumeItem{SAC-TIC \href{http://sac-tic.soneb.bj}{(sac-tic.soneb.bj)}}
%        {Permet d'être informé en cas de coupures ou de perturbations du réseau de distribution d’eau. $\rhd$ Développement du site de présentation du projet, développement backend (NodeJS + PostgreSQL), développement du formulaire de notification des bugs.}
        %\resumeItem{\href{https://solutis-it.com/}{solutis-it.com}}
          %{Conception et développement du site web de l'entreprise.}
%      \resumeItemListEnd

%    \resumeSubheading
%      {Développeur Junior}{Cotonou, Benin}
%      {EtriLabs \href{https://etrilabs.com/}{(etrilabs.com)}}{Apr 2017 - Sep 2017}
%      \resumeItemListStart
%        \resumeItem{Mongo Manager}
%          {Permet de backup vos bases de données MongoDB sur le stockage de votre choix: AWS S3, Digital Ocean,... $\rhd$ Développement backend NodeJs et Express.}
%      \resumeItemListEnd
      
      \resumeSubheading
      {Vice-responsable}{Cotonou, Benin}
      {Python Benin User Group}{Decembre 2018 - Août 2019}
      \resumeItemListStart
        \resumeItem{\href{https://pythonbenin.com/}{pythonbenin.com}}
          {Organisation de rencontres mensuelles autour de Django, Flask, Tensorflow...}
      \resumeItemListEnd

%    \resumeSubheading
%      {Google Developer Group Porto-Novo [Communauté]}{Porto-Novo, Benin}
%      {Co-lead \href{https://www.meetup.com/GDG-Porto-Novo/}{(meetup.com/GDG-Porto-Novo)}}{Jan 2018 - Jul 2019}
%      \resumeItemListStart
%	\resumeItem{Codelab: NodeJS Basics}
%          {Introduction a NodeJS. \href{https://docs.google.com/presentation/d/1MmO1npyTXAGGuoSzTBCVJkYp5vT-VT3UkC4ul2a9RHI/edit?usp=sharing}{\underline{Presentation}}.}

%        \resumeItem{Codelab: Tensorflow}
%          {Solve Real Problems with TensorFlow: From 0 to production ready app. \href{https://docs.google.com/presentation/d/1mA1svrJGFI6dQM-pmXnstHO20U28w70Qcd00fCanzFg/edit?usp=sharing}{\underline{Presentation}}.}
%      \resumeItemListEnd

  \resumeSubHeadingListEnd
  
  
%--------PROGRAMMING SKILLS------------
\section{\color{blue}Compétences Informatiques et Linguistiques}
\resumeSubHeadingListStart
 \item{
  	\textbf{Programmation}{: Python, C++, Matlab, Cuda, Code\_Aster, \LaTeX}
  	\hfill
  	\textbf{Langues}{: Anglais (Compréhension et rédaction de textes scientifiques)}
  }
\resumeSubHeadingListEnd
  


%-----------PROJECTS-----------------
\section{\color{blue}Compétences Scientifiques et Projets $\rightarrow$ github.com/kenn44}
%$\rhd$ Rapports et simulations
\resumeSubHeadingListStart
\resumeSubItem{Optimisation numérique et Simulation}

\resumeItemListStart
\item{Contrôle optimal de l'équation de Schrödinger avec l'algorithme du gradient à pas fixe et la méthode de splitting d'opérateur. Implémentation et simulation avec \textbf{Python}, NumPy et Matplotlib}

\item{Étude de modèles en dynamique des populations notamment ceux de Lotka-Volterra et Verhulst. (Scilab)}

\item{Optimisation non-linéaire sous contraintes par méthode SQP (projet: lanceur spatial en Matlab)}      

\resumeItemListEnd


\resumeSubItem{Modélisation et Analyse numérique}

\resumeItemListStart
\item{Approcher la solution d’une équation aux dérivées partielles via les méthodes \textbf{éléments finis} et \textbf{volumes finis} - Résolution d’un problème 2D elliptique en \textbf{C++}.}

\item{Redaction d'un notebook Jupyter présentant les principales méthodes numériques de résolution d'équations non linéaires $f(x)=0$ et leur implementation en \textbf{Python}.}

\item{Implémentation en Python de schémas numériques de résolution d'équations différentielles ordinaires	}      

\resumeItemListEnd

  
%\resumeSubItem{Peekaboo}{Application de chat vidéo. Développement d'une pipeline CI/CD pour automatiser les tests et déploiements.}
  
%    \resumeSubItem{AMMI Lightweight \href{https://suspicious-brown-5e9d1b.netlify.com/}{(Link)}}{Rebuild of the the African Master of Machine Intelligence website (aims-ammi.com) to improve the rendering performance. Repository: \href{https://github.com/kenn44/ammi-lightweight}{--github.com/kenn44/ammi-lightweight-- .}}
    
%\resumeSubItem{Peekaboo}
%\href{https://play.google.com/store/apps/details?id=mvp.peekaboo.dev}{(Play Store)}
%{Application de chat vidéo en temps réel. En charge du développement backend et responsable technique du projet, j'ai fait face à plusieurs challenges: configurer la pipeline CI/CD pour automatiser les tests et déploiements, implementer la communication temps reel avec WebSocket et WebRTC. J'ai pu travailler sur de nouvelles technologies, notamment Docker et essayer de nouvelles architectures: Serverless, GCP Compute Engine, AWS S3.}
      
%\resumeSubItem{Credit Scoring Model (Draft)}
%{A OpenSI \href{http://opensi.co/}{(opensi.co)}, nous travaillons sur l’utilisation de données alternatives afin de mieux évaluer la solvabilité dans le cadre de crédits. L'objectif final est d'améliorer l'accessibilité aux crédits des personnes exclues des établissements bancaires. Notes disponibles à \href{https://github.com/kenn44/credit-scoring101}{ (github.com/kenn44/credit-scoring101)} }

%\resumeSubItem{\href{blog.kennethassogba.co}{--blog.kennethassogba.co--} (In progress)}
%      {My personal blog.}

%\resumeSubItem{InoTech (Fermé) \href{https://inotech.ga/}{(inotech.ga)}}
%{Ino-tech est un média ambitieux fondé en 2017 pour étudier et montrer l'impact de la technologie sur nos vies.}
  \resumeSubHeadingListEnd




\section{\color{blue}Centres d'intérêt}
  \resumeSubHeadingListStart
    %\item{\textbf{Anglais}{: Opérationnel}}
	
	\item{
	\textbf{Top Aéro }{(top-aero.com): Vice responsable pôle aéronautique de l’association d’aéronautique et aérospatial de Sorbonne Université.}
	}
	\item{ \textbf{Exploration spatiale} habitée et non habitée: Passionné, j'ai suivi les lancements d'Ariane 5, de la navette spaciale Atlantis, l'épopée des sondes Voyager 1 et 2.
	}
  \resumeSubHeadingListEnd


%-------------------------------------------
\end{document}\grid
\grid
