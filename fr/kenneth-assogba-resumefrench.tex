%-------------------------
% Resume in Latex
% Author : Kenneth Assogba
% License : MIT
%------------------------

\documentclass[letterpaper,11pt]{article}

\usepackage{latexsym}
\usepackage[empty]{fullpage}
\usepackage{titlesec}
\usepackage{marvosym}
\usepackage[usenames,dvipsnames]{color}
\usepackage{verbatim}
\usepackage{enumitem}
\usepackage[hidelinks]{hyperref}
\usepackage{fancyhdr}
\usepackage[utf8]{inputenc}

\pagestyle{fancy}
\fancyhf{} % clear all header and footer fields
\fancyfoot{}
\renewcommand{\headrulewidth}{0pt}
\renewcommand{\footrulewidth}{0pt}

% Adjust margins
\addtolength{\oddsidemargin}{-0.5in}
\addtolength{\evensidemargin}{-0.5in}
\addtolength{\textwidth}{1in}
\addtolength{\topmargin}{-.5in}
\addtolength{\textheight}{1.0in}

\urlstyle{same}

\raggedbottom
\raggedright
\setlength{\tabcolsep}{0in}

% Sections formatting
\titleformat{\section}{
  \vspace{-4pt}\scshape\raggedright\large
}{}{0em}{}[\color{blue}\titlerule \vspace{-5pt}]

%-------------------------
% Custom commands
\newcommand{\resumeItem}[2]{
  \item\small{
    \textbf{#1}{: #2 \vspace{-2pt}}
  }
}

\newcommand{\resumeSubheading}[4]{
  \vspace{-1pt}\item
    \begin{tabular*}{0.97\textwidth}{l@{\extracolsep{\fill}}r}
      \textbf{#1} & #2 \\
      \textit{\small#3} & \textit{\small #4} \\
    \end{tabular*}\vspace{-5pt}
}

\newcommand{\resumeSubItem}[2]{\resumeItem{#1}{#2}\vspace{-4pt}}

\renewcommand{\labelitemii}{$\circ$}

\newcommand{\resumeSubHeadingListStart}{\begin{itemize}[leftmargin=*]}
\newcommand{\resumeSubHeadingListEnd}{\end{itemize}}
\newcommand{\resumeItemListStart}{\begin{itemize}}
\newcommand{\resumeItemListEnd}{\end{itemize}\vspace{-5pt}}

%-------------------------------------------
%%%%%%  CV STARTS HERE  %%%%%%%%%%%%%%%%%%%%%%%%%%%%


\begin{document}

%----------HEADING-----------------
\begin{tabular*}{\textwidth}{l@{\extracolsep{\fill}}r}
  \textbf{\href{https://kennethassogba.co/}{\Large Kenneth Assogba}} & {\color{blue} Email} : \href{mailto: kennethassogba@gmail.com}{kennethassogba@gmail.com}\\
  %\href{https://kennethassogba.co/}{https://kenneth.assogba.co} & Mobile : 06 14 26 95 55 \\
  Thèse en Analyse Numérique et Simulation & {\color{blue} Mobile} : 06 14 26 95 55 \\
  {\color{blue} Github} : \href{https://github.com/kenn44}{https://github.com/kenn44} & 25 ans (12/02/1995)
\end{tabular*}


%-----------EDUCATION-----------------
\section{\color{blue}Education}
  \resumeSubHeadingListStart
    \resumeSubheading
    %Calcul Scientifique
      {Master 2 Ingénierie Mathématique : Analyse Numérique \& Calcul Scientifique}{Paris, France}
      {\textbf{Sorbonne Université} (ex \textbf{Université Pierre et Marie Curie}) }{Sept. 2019 -- Present}
    \resumeSubheading
      {Master Mathématiques Fondamentales : EDP \& Géométrie}{Dangbo, Benin}
      {Institut de Mathematiques et de Sciences Physiques}{Oct. 2017 -- Août 2019}
    %\resumeSubheading
      %{Licence Informatique - Option Architecture Logicielle}{Cotonou, Benin}
      %{École Superieure de Gestion d'Informatique et des Sciences}{Oct. 2016 -- Dec. 2017}
    \resumeSubheading
      {Licence Mathematiques - Informatique}{Dangbo, Benin}
      {Institut de Mathematiques et de Sciences Physiques}{Oct. 2015 -- Jun. 2017}
    \resumeSubheading
      {Classes Préparatoires MPSI}{Dangbo, Benin}
      {Institut de Mathématiques et de Sciences Physiques}{Oct. 2013 -- Jun. 2015}
  \resumeSubHeadingListEnd


%-----------EXPERIENCE-----------------
\section{\color{blue}Experience}
  \resumeSubHeadingListStart
  
      \resumeSubheading
      {Stage de fin d'études}{Palaiseau, France}
      {\textbf{Total R\&D}}{Avril 2020 - Present}
      \resumeItemListStart
       \item{Hybrid mesh generation: from practical algorithms to discrete geometry challenges
      	\\$\rhd$ \small{Étude de la littérature sur le maillage hybride
      	\\$\rhd$ Découverte des bases de la géométrie algorithmique
      	\\$\rhd$ Implémentation d'un 1er algorithme en 2D avec l'utilisation de l'API Python de Gmsh}}
      \resumeItemListEnd
      
      \resumeSubheading
	  {Stagiare Assistant de Recherche}{Dangbo, Benin}
	  {Unité de Recherche en Mathématique et Physique Mathématique - IMSP}{Mai 2019 - Août 2019}
	  \resumeItemListStart
	  \item{Schémas monotones discrets pour l'équation de Schrödinger
	  \\{$\rhd$ \small{Étude de la littérature sur le contrôle optimal en mécanique quantique
	  \\$\rhd$ Construction de schémas monotones implicites et explicites
	  \\$\rhd$ Implémentation des algorithmes obtenus et simulations sous Octave}}}
	  \resumeItemListEnd

%    \resumeSubheading
%      {Artisan Développeur}{Cotonou, Benin}
%      {Solutis \href{https://solutis-it.com/}{(solutis-it.com)} }{Oct 2017 - Mars 2019}
%      \resumeItemListStart
%        \resumeItem{SAC-TIC \href{http://sac-tic.soneb.bj}{(sac-tic.soneb.bj)}}
%        {Permet d'être informé en cas de coupures ou de perturbations du réseau de distribution d’eau. $\rhd$ Développement du site de présentation du projet, développement backend (NodeJS + PostgreSQL), développement du formulaire de notification des bugs.}
        %\resumeItem{\href{https://solutis-it.com/}{solutis-it.com}}
          %{Conception et développement du site web de l'entreprise.}
%      \resumeItemListEnd

%    \resumeSubheading
%      {Développeur Junior}{Cotonou, Benin}
%      {EtriLabs \href{https://etrilabs.com/}{(etrilabs.com)}}{Apr 2017 - Sep 2017}
%      \resumeItemListStart
%        \resumeItem{Mongo Manager}
%          {Permet de backup vos bases de données MongoDB sur le stockage de votre choix: AWS S3, Digital Ocean,... $\rhd$ Développement backend NodeJs et Express.}
%      \resumeItemListEnd
      
      %\resumeSubheading
      %{Vice-responsable}{Cotonou, Benin}
      %{Python Benin User Group}{Decembre 2018 - Août 2019}
      %\resumeItemListStart
      %  \resumeItem{\href{https://pythonbenin.com/}{pythonbenin.com}}
      %    {Organisation de rencontres mensuelles autour de Django, Flask, Tensorflow...}
      %\resumeItemListEnd

%    \resumeSubheading
%      {Google Developer Group Porto-Novo [Communauté]}{Porto-Novo, Benin}
%      {Co-lead \href{https://www.meetup.com/GDG-Porto-Novo/}{(meetup.com/GDG-Porto-Novo)}}{Jan 2018 - Jul 2019}
%      \resumeItemListStart
%	\resumeItem{Codelab: NodeJS Basics}
%          {Introduction a NodeJS. \href{https://docs.google.com/presentation/d/1MmO1npyTXAGGuoSzTBCVJkYp5vT-VT3UkC4ul2a9RHI/edit?usp=sharing}{\underline{Presentation}}.}

%        \resumeItem{Codelab: Tensorflow}
%          {Solve Real Problems with TensorFlow: From 0 to production ready app. \href{https://docs.google.com/presentation/d/1mA1svrJGFI6dQM-pmXnstHO20U28w70Qcd00fCanzFg/edit?usp=sharing}{\underline{Presentation}}.}
%      \resumeItemListEnd

  \resumeSubHeadingListEnd
  
  
%--------PROGRAMMING SKILLS------------
\section{\color{blue}Compétences Informatiques et Linguistiques}
\resumeSubHeadingListStart
 \item{
  	\textbf{Programmation}{: Python, C++, Code\_Aster (Salome\_Meca), Matlab, Gmsh, Freefem++, Git, \LaTeX}
  }
 \item{
	\textbf{Langues}{: Anglais (Compréhension et rédaction de textes scientifiques)}
	}
\resumeSubHeadingListEnd
  


%-----------PROJECTS-----------------
\section{\color{blue}Compétences Scientifiques et Projets}
%$\rightarrow$ github.com/kenn44
%$\rhd$ Rapports et simulations
\resumeSubHeadingListStart


\resumeSubItem{Calcul scientifique et Optimisation numérique}

\resumeItemListStart

\item{Étude du conditionnement et de l'influence du choix du solveur (MUMPS, GCPC, MULT\_FRONT) dans la résolution d'un problème de mécanique ou de thermique dans \textbf{Salome\_Meca}.}

\item{Parallélisation de la résolution d'un système avec la méthode du gradient conjugué (\textbf{MPI}).}

\item{Contrôle optimal de l'équation de Schrödinger avec l'algorithme du gradient à pas fixe (\textbf{Python} et NumPy).}
%\item{Contrôle optimal de l'équation de Schrödinger avec l'algorithme du gradient à pas fixe et la méthode de splitting d'opérateur. Implémentation et simulation avec \textbf{Python}, NumPy et Matplotlib}

\item{Optimisation non-linéaire sous contraintes par méthode SQP (projet: lanceur spatial en Matlab)}

%\item{Étude de modèles en dynamique des populations notamment ceux de Lotka-Volterra et Verhulst. (Scilab)}

     

\resumeItemListEnd



\resumeSubItem{Modélisation et Analyse numérique}

\resumeItemListStart
\item{Modéliser un problème scientifique: Étude de la dissipation thermique d'un rayonnement laser dans la peau.}

\item{Implémentation en \textbf{C++} un algorithme de recherche d’un triangle $K$ dans un maillage convexe $T_h$ contenant un point $(x, y)$ en $O(log_2(n_T))$.}

\item{Rédaction d'un notebook Jupyter présentant les méthodes numériques de résolution d'équations non linéaires $f(x)=0$ et leur implementation en \textbf{Python}.}

\item{Implémentation en Python de schémas numériques de résolution d'équations différentielles ordinaires	}      

\resumeItemListEnd




  
%\resumeSubItem{Peekaboo}{Application de chat vidéo. Développement d'une pipeline CI/CD pour automatiser les tests et déploiements.}

  \resumeSubHeadingListEnd




\section{\color{blue}Centres d'intérêt}
  \resumeSubHeadingListStart
    %\item{\textbf{Anglais}{: Opérationnel}}
	
	\item{
	\textbf{Top Aéro }{(top-aero.com): Vice responsable pôle aéronautique de l’association d’aéronautique et aérospatial de Sorbonne Université.}}

	%\item{ \textbf{Exploration spatiale} habitée et non habitée: Passionné, j'ai suivi les lancements d'Ariane 5, de la navette spaciale Atlantis, l'épopée des sondes Voyager 1 et 2.}
  \resumeSubHeadingListEnd


%-------------------------------------------
\end{document}\grid
\grid
