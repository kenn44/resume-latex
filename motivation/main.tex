\documentclass[11pt,a4paper,sans]{moderncv}        % possible options include font size ('10pt', '11pt' and '12pt'), paper size ('a4paper', 'letterpaper', 'a5paper', 'legalpaper', 'executivepaper' and 'landscape') and font family ('sans' and 'roman')

% moderncv themes
\moderncvstyle{classic}                            % style options are 'casual' (default), 'classic', 'oldstyle' and 'banking'
\moderncvcolor{green}                              % color options 'blue' (default), 'orange', 'green', 'red', 'purple', 'grey' and 'black'
%\renewcommand{\familydefault}{\sfdefault}         % to set the default font; use '\sfdefault' for the default sans serif font, '\rmdefault' for the default roman one, or any tex font name
%\nopagenumbers{}                                  % uncomment to suppress automatic page numbering for CVs longer than one page

% character encoding
\usepackage[utf8]{inputenc}                       % if you are not using xelatex ou lualatex, replace by the encoding you are using
%\usepackage{CJKutf8}                              % if you need to use CJK to typeset your resume in Chinese, Japanese or Korean

% adjust the page margins
\usepackage[scale=0.75]{geometry}
%\setlength{\hintscolumnwidth}{3cm}                % if you want to change the width of the column with the dates
%\setlength{\makecvtitlenamewidth}{10cm}           % for the 'classic' style, if you want to force the width allocated to your name and avoid line breaks. be careful though, the length is normally calculated to avoid any overlap with your personal info; use this at your own typographical risks...

% personal data
\name{Kenneth}{Assogba}
\title{Candidature}                               % optional, remove / comment the line if not wanted
\address{17, Rue Pierre Semard}{94310 Orly}{France}% optional, remove / comment the line if not wanted; the "postcode city" and and "country" arguments can be omitted or provided empty
\phone[mobile]{06~14~26~95~55}                   % optional, remove / comment the line if not wanted
%\phone[fixed]{+2~(345)~678~901}                    % optional, remove / comment the line if not wanted
%\phone[fax]{+3~(456)~789~012}                      % optional, remove / comment the line if not wanted
\email{kenneth.assogba@etu.upmc.fr}                               % optional, remove / comment the line if not wanted
%\homepage{www.johndoe.com}                         % optional, remove / comment the line if not wanted
%\extrainfo{additional information}                 % optional, remove / comment the line if not wanted
%\photo[64pt][0.4pt]{picture}                       % optional, remove / comment the line if not wanted; '64pt' is the height the picture must be resized to, 0.4pt is the thickness of the frame around it (put it to 0pt for no frame) and 'picture' is the name of the picture file
%\quote{Some quote}                                 % optional, remove / comment the line if not wanted

% to show numerical labels in the bibliography (default is to show no labels); only useful if you make citations in your resume
%\makeatletter
%\renewcommand*{\bibliographyitemlabel}{\@biblabel{\arabic{enumiv}}}
%\makeatother
%\renewcommand*{\bibliographyitemlabel}{[\arabic{enumiv}]}% CONSIDER REPLACING THE ABOVE BY THIS

% bibliography with mutiple entries
%\usepackage{multibib}
%\newcites{book,misc}{{Books},{Others}}
%----------------------------------------------------------------------------------
%            content
%----------------------------------------------------------------------------------
\begin{document}
%-----       letter       ---------------------------------------------------------
% recipient data
\recipient{Aux responsables du recrutement}{
	Framatome\\
	Direction Technique et Ingénierie\\
	92 - La Défense}
\date{Samedi, 16 Novembre 2019}
\opening{Mesdames, Messieurs,}
\closing{Veuillez agréer, Mesdames, Messieurs, l’expression de mes salutations distinguées,}
%\enclosure[Attached]{curriculum vit\ae{}}          % use an optional argument to use a string other than "Enclosure", or redefine \enclname
\makelettertitle

\textbf{Objet :} Stage Ingénieur Analyse \& Optimisation Codes de Calculs Neutronique-Réseaux

Actuellement étudiant en Master 2 Ingénierie Mathématique - Analyse Numérique, Calcul Scientifique option Mécanique à Sorbonne Université (ex Université Pierre et Marie Curie), je suis particulièrement intéressé par le sujet : "Analyse \& Optimisation Codes de Calculs Neutronique-Réseaux" que vous proposez.

Framatome, leader mondial dans la conception, la réalisation et l'amélioration des réacteurs nucléaires, brille depuis plusieurs années par son activité de Recherche et Développement. Framatome mène ou participe à plusieurs projets de recherche dans le domaine du nucléaire notamment le projet ITER sur la fusion nucléaire ou encore l’optimisation et la sûreté des réacteurs nucléaires. Ce dernier pôle de recherche m'intéresse réellement.

En effet entre mai et août 2019, j’ai effectué un stage de recherche sur les Schémas monotones discrets pour l'équation de Schrödinger. Ce stage m'a permis de travailler sur la construction de nouvelles méthodes numériques pour le contrôle optimal en mécanique quantique. Cette première expérience à la frontière entre sciences fondamentales et appliquées m’a donné le goût pour la recherche en modélisation et analyse numérique.
J’ai également pu acquérir au cours de celui-ci, la rigueur, la créativité et l’autonomie nécessaire à la réussite du projet que vous proposez.

Ce stage à Framatome est donc définitivement, l'opportunité, pour moi de confronter directement les connaissances théoriques acquises au monde de l’industrie nucléaire et de satisfaire mon appétit de connaissance.





\makeletterclosing

\end{document}


%% end of file `template.tex'.
%% Copyright 2006-2013 Xavier Danaux (xdanaux@gmail.com).
