%% This is a LaTeX template for preparing papers for Publ. Inst. Math.; version January 2016
%% Please delete everything begining with %% (DOUBLE %).

% Submission number: 
\documentclass[a4paper,draft]{amsproc}
\usepackage{amssymb}
\usepackage[hyphens]{url} \urlstyle{same}
%\usepackage[dvips]{graphicx} %% Package for inserting illustrations/figures

%% The following packages are useful (you may want to use them):
%\usepackage{refcheck} %% Checks whether enumerated equations are referred to or not.
                       %% Please remove unnecessary numbers.
%\usepackage{cmdtrack} %% Checks whether all author defined macros are used or not
                       %% (see the end of .log file); unused ones should be removed.
%% Both packages have limitations---consult the package documentation.

\theoremstyle{plain}
 \newtheorem{thm}{Theorem}[section]
 \newtheorem{prop}{Proposition}[section]
 \newtheorem{lem}{Lemma}[section]
 \newtheorem{cor}{Corollary}[section]
\theoremstyle{definition}
 \newtheorem{exm}{Example}[section]
 \newtheorem{dfn}{Definition}[section]
\theoremstyle{remark}
 \newtheorem{rem}{Remark}[section]
 \numberwithin{equation}{section}

%% Please, do not change the following four lines:
\renewcommand{\le}{\leqslant}\renewcommand{\leq}{\leqslant}
\renewcommand{\ge}{\geqslant}\renewcommand{\geq}{\geqslant}
\renewcommand{\setminus}{\smallsetminus}
\setlength{\textwidth}{28cc} \setlength{\textheight}{42cc}

\title[Running title / header]{TITLE}

\subjclass[2010]{Primary REQUIRED; Secondary OPTIONAL}

%% Please use the newest classification -- 2010
%% available at  http://msc2010.org/MSC-2010-server.html
%% and the newest amsproc.cls.
%% Please, classify to the third level,
%% e.g., 26A and 26Axx are not satisfsctory.

\keywords{optional, but desirable}

\author[Surname]{\bfseries Name Surname} %% Please write ful names, avoid initials

\address{ %% Put here your affiliation; street address is not required
Department of Mathematics \\ % \hfill (Received 00 00 201?)\\
Our University   \\ %\hfill (Revised  00 00 201?)\\
Town\\
Country}
\email{user@server}

%% OTHER AUTHOR(S):
%\author[]{}
%\address{ }
%\email{}

\thanks{Supported by ... } %% optional
\thanks{Communicated by ...} %% This will be filled in the journal office.

\begin{document}

{\begin{flushleft}\baselineskip9pt\scriptsize
%PUBLICATIONS DE L'INSTITUT MATH\'EMATIQUE\newline
%Nouvelle s\'erie, tome ??(1??)) (201?), od--do \hfill DOI: \\
MANUSCRIPT
\end{flushleft}}
\vspace{18mm} \setcounter{page}{1} \thispagestyle{empty}


\begin{abstract}
An abstract is REQUIRED!
Please do not use author defined macros in the abstract
and avoid references to anything in the paper,
since the abstract will be detached from the article.
\end{abstract}

\maketitle

\section{Section title}  %% Please avoid complex formulas in (sub)titles

This template contains detailed instructions for preparing manuscripts for our journal.
Missing to follow them may cause rejecting of the manuscript without further processing.
So, please read both the source and compiled texts.

Insert your text here. Make sure that it is written in correct English. 
The guides \cite{TrzD,TrzE} may help you with it.

If there are subsections, then you may use

\subsection{Subsection title}
You may also use subsubsections,
but please put a line or two of text between the subsection
and the subsubsection titles.

Proclaims (theorems, propositions,...) should be inserted as follows:

\begin{thm} \label{some label} % of course, label is optional
Statement of the theorem.
\end{thm}

Please, do not put a proclaim immediately after a subtitle of any level.
Write a line or two of text in between.

\begin{proof}
Your proof.
Please do not use the quantifiers $\forall,\exists$ as abbreviations,
i.e., use them only in the papers from formal logics.
The symbol for the end of the proof will appear automatically.
\end{proof}

For displayed equations (formulas) you may use
\begin{equation}\label{eq:a1b}
e^{i\pi}=-1
\end{equation}
and/or similar \LaTeX\ constructions (align(ed), multline, gather(ed),\dots).
That way, you may refer to \eqref{eq:a1b} in the subsequent text.
We strongly encourage the usage of this dynamic system of referencing
instead of explicitly writing, for example, (1.1).

If you do not refer to an equation, then you may write it as
\[
e^{i\pi}=-1
\]
(preferred) or
\begin{equation*}
e^{i\pi}=-1
\end{equation*}
In such a starred version the equation will not be numbered.
If you want to use a distinctive tag to an equation,
you may do that in the following manner:
\begin{equation}\label{dist}
e^{i\pi}=-1
\tag{*}
\end{equation}
So you can refer to \eqref{dist}.

Formulas should be displayed \emph{only}
if they must be numbered for a subsequent reference
or if they are too long or complicated.
Please \emph{do not} number displayed formulas that are not referred to.

Send figures/illustrations as eps files (each figure/illustration in a separate file).
They can be inserted in the following way:

\begin{figure}[htb]
%\includegraphisc[width=99mm]{filename.eps}
\caption{}
\label{some label}
\end{figure}

Besides the standard handbooks on \LaTeX\ \cite{Gr,Lmp,Lbible},
please consult the short and useful guide \cite{TrzG}.


\section{List of references}

The list of references should be written as below.

Only \emph{standard} abbreviations for names of journals and other serials
should be used (see \url{http://zbmath.org/journals/}).


\bibliographystyle{amsplain}
\begin{thebibliography}{n} %% n is number of items, or the largest label

\bibitem{Exm_paper} A.\,U. Thor, (not Thor, A.U.!)
\emph{Title of paper},
J. Math. \textbf{99} (2008), 111--222.

\bibitem{Exm_in_book} A.\,U. Thor,
\emph{Title of paper},
in: E. Ditor (ed.), \emph{Title of Book}, Publisher, City, Year, 888--999.

\bibitem{Gr} G. Gr\"atzer,
\emph{More Math Into \LaTeX}, 4th ed.,
Springer, 2007.

\bibitem{Lmp} L. Lamport,
\emph{\LaTeX: A Document Preparation System}, 2nd ed.,
Addison-Wesley, 1994.

\bibitem{Lbible} F. Mittelbach,  M. Goossens (with J. Braams, D. Carlisle, C. Rowley),
\emph{The \LaTeX\ Companion}, 2nd ed.,
Addison-Wesley, 2004.

\bibitem{TrzG} J. Trzeciak, \emph{Writing mathematical papers---a few tips}, available at\\
\url{https://www.impan.pl/wydawnictwa/dla-autorow/writing.pdf}

\bibitem{TrzE} J. Trzeciak, \emph{Writing Mathematical Papers in English},
European Mathematical Society, Zurich, 2005; a version available at
\url{https://libgen.unblocked.zone/book/1043688}

\bibitem{TrzD} J. Trzeciak, \emph{Mathematical English Usage. A Dictionary}, available at\\
\url{http://www.impan.pl/Dictionary}

\end{thebibliography}

\end{document}

%% To be filled in the journal office:

@author:
@affiliation:
@title:
@language: English
@pages:
@classification1:
@classification2:
@keywords:
@abstract:
@filename:
@EOI


\grid
